\documentclass[colorhighlight]{bidipresentation}
\usepackage{amsmath}
\usepackage{graphicx}
\definergbcolor{textcolor}{000000}%
\definergbcolor{inactivecolor}{B2B2B1}%
\usepackage{xepersian}
\settextfont{XB Niloofar}
\setlatintextfont{Times New Roman}
\begin{document}
\begin{rawslide}
\tableofcontents
\end{rawslide}
\begin{rawslide}
\section{لیست‌ها}
\begin{flipitem}
\item 
این متنی کمی
\item
این متن طولانی است که بیشتر از یک سطر باشد تا ببینم مشکلی با رنگ وجود دارد یا خیر و ادامه متن که ادامه پیدا می‌کند در سطر بعدی
\item سه
\end{flipitem}
\begin{flipenum}
\item 
این متنی کمی
\item
این متن طولانی است که بیشتر از یک سطر باشد تا ببینم مشکلی با رنگ وجود دارد یا خیر و ادامه متن که ادامه پیدا می‌کند در سطر بعدی
\item سه
\end{flipenum}
\end{rawslide}
\begin{rawslide}
\section{جدول}
  \newcommand{\tabend}{\\\hline}%

  \liststepwise{%
    \begin{center}
      \step{%
        \begin{tabular}{|r|r|r|}
          \hline
        1 & 2 & 3%
          \step{\\\hline 4\renewcommand{\tabend}{\\\cline{1-1}}}%
          \step{& 5\renewcommand{\tabend}{\\\cline{1-2}}}%
          \step{& 6\renewcommand{\tabend}{\\\hline}}%
          \step
          {%
            \\\hline
                                %
                                % Again, \step's are nested inside each other...
                                %
            \step{7}&\step{8}&\step{9}%
            }%
          \step{\\\hline اما\renewcommand{\tabend}{\\\cline{1-1}}}%
          \step{&مواظب\renewcommand{\tabend}{\\\cline{1-2}}}%
          \step{& متن بلند باشید!\renewcommand{\tabend}{\\\hline}}%
          \tabend
        \end{tabular}%
        }%
    \end{center}%
    }%
\end{rawslide}
\begin{rawslide}
\section{فرمول چندخطی}
\liststepwise%
{%
  %
  % This is just for compressing the equations so they can be squeezed on one slide.
  %
  \fontsize{7.8pt}{9pt}\selectfont
  \renewcommand{\arraystretch}{0}%
  \setlength{\arraycolsep}{0pt}%
  \setlength{\abovedisplayskip}{0pt}%
  \setlength{\belowdisplayskip}{0pt}%
  %
  % \highlightboxed will be used for underlaying some formulas with color. To minimize overlap, the width of the outer
  % frame is reduced.
  \setlength{\highlightboxsep}{1pt}%
  %
  \begin{align*}
    \lefteqn
    {%
      \min
      \left(
        % The nested braces are filled `from outer to inner'. This means nesting a lot of steps inside each other...
        % The outermost brace is displayed from the outset.
        % The first step (which follows right here) displays the next inner brace (the first argument of \min), filled
        % with an almost `empty' array (apart from one comma and some dots).
        % \bstep is used to get appropriate white space when the step is not yet active.
        \bstep
        {\max
          \left(
            \begin{array}{l}
              % The next two steps fill in the lines of the array.
              \bstep{\min\left(F'(x),\min\left(F_1(x),G_1(y)\right)\right)},\\[-2ex]
              \vdots\\
              \bstep{\min\left(F'(x),\min\left(F_n(x),G_n(y)\right)\right)}
            \end{array}
          \right)
          },
        % After the first brace is filled, the next step provides the second argument of \min.
        \bstep{\min\left(G_i(y),H_i(z)\right)}
      \right)
      }
    &
    % The next couple of steps will create the remaining lines of the aligned equations. These need to be
    % insubstantial (as is the default for \liststepwise), because & can't go in a box.
    % As a consequence, the horizontal alignment cannot kick in until the last step is performed. This would make the
    % alignment `flicker' sidewise.
    % So we have to bite the bullet and duplicate the widest entry here (invisibly), so that the horizontal alignment
    % is constant during all steps. *sigh*
    \phantom
    {%
      {}=
      \min
      \left(
        F'(x),
        \min
        \left(
          \max
          \left(
            \begin{array}{l}
              \min\left(F_1(x),\min\left(G_1(y),G_i(y)\right)\right),\\[-1.5ex]
              \vdots\\[-.5ex]
              \min\left(F_n(x),\min\left(G_n(y),G_i(y)\right)\right)
            \end{array}
          \right),
          H_i(z)
        \right)
      \right)
      }
    % The next step displays two lines at a time, but incompletely, i.e. some parts are missing (which are inside
    % nested calls of \bstep).
    % This way, it is demonstrated how the arguments of the nested \min's are reordered.
    \step
    {%
      \\
      &=
      \max
      \left(
        % The macro \activatestep is used by \stepwise to `wrap' the argument of a \bstep command at the _first_ time
        % it appears.
        % Usually, it does nothing. Now, we redefine it to highlight its background, so it is easier to spot the
        % places where the additional arguments were inserted.
        \let\activatestep\highlightboxed
        \begin{array}{l}
          \min
          \left(
            % The inner \bstep's display the missing arguments, which are completely identical in both lines.
            % It is intended that all the missing arguments appear at the same time, so \rebstep is used for the
            % remaining arguments which have been left out.
            \min\left(\bstep{F'(x)},\min\left(\rebstep{F_1(x),G_1(y)}\right)\right),\min\left(G_i(y),H_i(z)\right)
          \right),\\[-2ex]
          \vdots\\[-1ex]
          \min
          \left(
            \min\left(\rebstep{F'(x)},\min\left(\rebstep{F_n(x),G_n(y)}\right)\right),\min\left(G_i(y),H_i(z)\right)
          \right)
        \end{array}
      \right)
      \\
      &=
      \max
      \left(
        \let\activatestep\highlightboxed
        \begin{array}{l}
          \min
          \left(
            \min\left(
              % Here are the remaining arguments of \min which are all to be displayed in one step (together with
              % those from the previous line).
              \rebstep{F'(x)},\min\left(\rebstep{F_1(x)},\min\left(\rebstep{G_1(y)},G_i(y)\right)\right)
            \right),
            H_i(z)
          \right),\\[-2.5ex]
          \vdots\\[-1.5ex]
          \min
          \left(
            \min\left(
              \rebstep{F'(x)},\min\left(\rebstep{F_n(x)},\min\left(\rebstep{G_n(y)},G_i(y)\right)\right)
            \right),
            H_i(z)
          \right)
        \end{array}
      \right)
      }
    \step
    {%
      \\
      &=
      \min
      \left(
        F'(x),
        \min
        \left(
          \max
          \left(
            \begin{array}{l}
              \min\left(F_1(x),\min\left(G_1(y),G_i(y)\right)\right),\\[-1.5ex]
              \vdots\\[-.5ex]
              \min\left(F_n(x),\min\left(G_n(y),G_i(y)\right)\right)
            \end{array}
          \right),
          H_i(z)
        \right)
      \right)
      }
  \end{align*}
  }%
\end{rawslide}
\begin{rawslide}
\section{شکل}
\begin{LTR}
\begin{center}%
  \stepwise
  {%
    \setlength{\unitlength}{.95cm}%
    \delimitershortfall-1sp% Just for the nested braces
    \begin{picture}(14,2)
      \put(0,1){\vector(1,0){1}}
      \put(0.5,0.5){\makebox(0,0){\small $x(t)$}}
      \put(13,1){\vector(1,0){1}}
      \put(13.5,0.5){\makebox(0,0){\small $y(t)$}}
      \step
      {
        \put(1,1){\line(3,2){1.5}}
        \put(1,1){\line(3,-2){1.5}}
        \put(2.5,0){\line(0,1){2}}
        \put(2,1){\makebox(0,0){\large $\varphi$}}
        }
      \step
      {
        \put(2.5,1){\vector(1,0){3.5}}
        \put(4.25,0.5){\makebox(0,0){\small $F_t = \varphi\left(x(t)\right)$}}
        }
      \step
      {
        \put(6,0){\framebox(2,2){\large $\Phi$}}
        }
      \step
      {
        \put(8,1){\vector(1,0){3.5}}
        %
        % Here, we find another nested use of \step inside \step.
        % \bstep is a variant of \step which _always_ puts its argument into a box for leaving the correct amount of
        % white space. We cannot use \parstepwise here because \put can't go into a box. Hence, just using \step for
        % building the nested formula on the next line would give the wrong size for the nested braces.
        %
        \put(9.75,0.5){\makebox(0,0){\small $G_t = \Phi\left(\bstep{\varphi\left(\bstep{x(t)}\right)}\right)$}}
        }
      \step
      {
        \put(13,1){\line(-3,2){1.5}}
        \put(13,1){\line(-3,-2){1.5}}
        \put(11.5,0){\line(0,1){2}}
        \put(12,1){\makebox(0,0){\large $\delta$}}
        }
    \end{picture}%
    }%
\end{center}%
\end{LTR}
\end{rawslide}
\begin{rawslide}
\section{پر کردن جای خالی}
\newcommand{\placeholder}[1]{\leavevmode\phantom{#1}\llap{\rule{\widthof{\phantom{#1}}}{\fboxrule}}}%
  %
  % We use the custom command \parstepwise which not only wraps the whole argument of \stepwise into a minipage (because
  % otherwise vertical spacing goes haywire, don't ask me why), but also gives substance to steps.
  %
  % All variants of \stepwise take an optional argument the contents of which are executed inside a group before the
  % inner loop of starts. It can be used to set parameters locally.
  % Here, we redefine \activatestep (which has been explained in the equation example) to highlight the first
  % appearance of any word.
  % \hidestepcontents is used as a `wrapper' for those arguments of \step which should not appear yet. It either
  % displays nothing (this is the default for \stepwise and \liststepwise) or puts its argument into a \phantom
  % (the default for \parstepwise); this behaviour is also toggled by \boxedsteps and \nonboxedsteps.
  % Here, we redefine it to use our selfdefined \placeholder to mark `missing' words.
  %
  \parstepwise[\let\hidestepcontents=\placeholder\let\activatestep=\highlightboxed]%
  {%
    \begin{quote}
      \Huge ما می‌تونیم \step{جای خالی} را \step{پر کنیم}
      %
      % \step takes an optional argument with which it can be specified _when_ its argument is to appear. This is
      % expressed in \ifthenelse syntax (see the documentation of the ifthen package).
      % Here, we refer to the counter step which is advanced by \stepwise and contains the number of the current step.
      % This way, steps can be made to appear in any order.
      \step[\value{step}=5]{متن} که در  ایجا
      \step[\value{step}=4]{پر شده} و      \step[\value{step}=3]{\textbf{هر}} ترتیبی داره!
    \end{quote}
    }%
\end{rawslide}
\begin{rawslide}
\section{نوشتن متن در جهت برعکس}
  \newcounter{reversestepno}%
  \newcommand{\backstep}{\step(\setcounter{reversestepno}{\value{totalsteps}-\value{stepcommand}+1}\ifthenelse{\value{step}=\value{reversestepno}})}%
  %
  % We use the custom command \parstepwise which not only wraps the whole argument of \stepwise into a minipage (because
  % otherwise vertical spacing goes haywire, don't ask me why), but also gives substance to steps.
  %
  % If the following \stepwise command would only contain the calls to \backstep, everything would be fine.
  % But we _had_ to add something else....
  % In the second part of this application of \stepwise, several steps are executed simultaneously with those executed
  % backwards in the first part. This means the value of the counter totalsteps is 14, i.e. the calls to \backstep
  % correspond to steps 8...14. To remedy this, we decree that the first step performed shall be number 7, by setting
  % the counter firststep accordingly in the optional argument of \stepwise.
  %
  \parstepwise[\setcounter{firststep}{\value{totalsteps}/2+\value{firststep}}]
  {%
    \begin{quote}
      \Huge
      \backstep{آیا} \backstep{الان} \backstep{می‌توان}
      \backstep{متن را} \backstep{در جهت} \backstep{برعکس}
      \backstep{نوشت\,!}

      \bigskip

      % By determining explicitly the times at which the following steps are executed, we make them appear
      % simultaneously with the preceding flock of \backsteps. As we have set the counter firststep to 7, we start
      % counting with 8.
      %
      \step[\value{step}=8]{\includegraphics[width=2cm]{fig-1}}
      \step[\value{step}=9]{\includegraphics[width=2cm]{fig-1}}
      \step[\value{step}=10]{\includegraphics[width=2cm]{fig-1}}
      \step[\value{step}=11]{\includegraphics[width=2cm]{fig-1}}
      \step[\value{step}=12]{\includegraphics[width=2cm]{fig-1}}
      \step[\value{step}=13]{\includegraphics[width=2cm]{fig-1}}
      \step[\value{step}=14]{\includegraphics[width=2cm]{fig-1}}%
    \end{quote}
    }%
\end{rawslide}
\end{document}